\documentclass[oneside]{book}

\usepackage[utf8]{inputenc}
\usepackage[T2A]{fontenc}
\usepackage[russian]{babel}
\usepackage[left = 0.3\textwidth, right = 0.3\textwidth]{geometry}
\usepackage{parskip}
\usepackage[fleqn]{amsmath}
\usepackage{mathtools}
\usepackage{amsfonts}
\usepackage{amssymb}
\usepackage{graphicx}
\usepackage{hyperref}
\usepackage{bookmark}
\usepackage{textcomp}

\setlength{\parskip}{0.03\textheight}

\graphicspath{{images/}}

\hypersetup{
    colorlinks,
    citecolor=black,
    filecolor=black,
    linkcolor=black,
    urlcolor=black
}

\newcommand{\meta}[1]{\text{<}#1\text{>}}
\newcommand{\set}[1]{\left\{#1\right\}}

\title{Логика: \\ упражнения}
\date{\today}
\author{Мы}

\begin{document}
    \maketitle

    \section{Дашков ЕВ | 1.2.12}
    Приведите к противоречию предположение о существовании множества всех множеств.

    \begin{flalign*}
        &] \ \exists x \ \forall y \in x
        \implies
        \exists x \in x \ \ \
        \begin{gathered}
            \implies \\
            \exists \set{x, y}
        \end{gathered} \ \ \
        \exists x, \set{x, x} \ x \in x
        \implies
        \exists x, \set{x} \ x \in x \\
        &\begin{gathered}
            \implies \\
            \exists y \in x \ \forall z \in z \ z \not\in y
        \end{gathered} \ \ \
        \exists x, \set{x} \
        \begin{cases}
            x \in x \\
            \exists w \in \set{x} \ \forall i \in \set{x} \ i \not\in \set{x}
        \end{cases}
        \implies \\
        &\exists x, \set{x} \
        \begin{cases}
            x \in x \\
            \forall i \in \set{x} \ i \not\in x
        \end{cases}
        \begin{gathered}
            \iff \\
            ?
        \end{gathered} \ \ \
        \exists x, \set{x} \
        \begin{cases}
            x \in x \\
            x \not\in x
        \end{cases}
        \implies
        \bot \\
        &\overline{\exists x \ \forall y \in x}
    \end{flalign*}
    Можно попробовать решить без аксиомы фундированности.

    \section{Дашков ЕВ | 1.2.13}
    Объясните, почему степень множества $ A $ единственна.

    \begin{flalign*}
        &] \ \exists x, y \ \forall z \
        \begin{cases}
            z \in x \iff z \subseteq A \\
            z \in y \iff z \subseteq A \\
            x \neq y
        \end{cases}
        \begin{gathered}
            \iff \\
            \overline{def \ =}
        \end{gathered} \ \ \
        \exists x, y \
        \begin{cases}
            \forall z \ (z \in x \iff z \subseteq A) \\
            \forall z \ (z \in y \iff z \subseteq A) \\
            \exists z \ (z \in x \oplus z \in y)
        \end{cases}
        \iff \\
        &\begin{cases}
            \forall z \ (z \in x \iff z \subseteq A) \\
            \forall z \ (z \in y \iff z \subseteq A) \\
            \exists z \ (z \subseteq A \oplus z \subseteq A)
        \end{cases}
        \implies
        \bot \\
        &\exists! \mathcal{P}(A)
    \end{flalign*}

    \section{Дашков ЕВ | 1.2.15}
    Выпишите все элементы множества
    $ \mathcal{P}\left(\set{\varnothing, \set{\varnothing}}\right) $.

    \begin{flalign*}
        \mathcal{P}\left(\set{\varnothing, \set{\varnothing}}\right)
        =
        \set {
        \varnothing,
        \set{\varnothing},
        \set{\set{\varnothing}},
        \set{\varnothing, \set{\varnothing}}
        }
    \end{flalign*}

    \section{Дашков ЕВ | 1.2.19}
    Докажите, что $ \cup\varnothing = \varnothing $ и $ \cup\set{A} = A $
    для всех $ A $.

    \begin{flalign*}
        &\left(x \in A \iff x \in A\right)
        \iff
        \left(
        \exists z \
        \begin{cases}
            z = A \\
            x \in z
        \end{cases}
        \iff x \in A
        \right)
        \begin{gathered}
            \iff \\
            ?
        \end{gathered} \\
        &\begin{cases}
            \exists y \
            \begin{cases}
                y \in \varnothing \\
                x \in y
            \end{cases}
            \iff x \in \varnothing \\
            \exists z \
            \begin{cases}
                z \in \set{A} \\
                x \in z
            \end{cases}
            \iff x \in A
        \end{cases}
        \begin{gathered}
            \iff \\
            def \ \cup
        \end{gathered}
        \begin{cases}
            x \in \cup\varnothing \iff x \in \varnothing \\
            x \in \cup\set{A} \iff x \in A
        \end{cases}
        \begin{gathered}
            \iff \\
            def =
        \end{gathered} \\
        &\begin{cases}
            \cup\varnothing = \varnothing \\
            \cup\set{A} = A
        \end{cases}
    \end{flalign*}

    \section{Дашков ЕВ | 1.2.20}
    Докажите, что если $ X \subseteq Y $, то $ \cup X \subseteq \cup Y $
    для любых множеств $ X $ и $ Y $.

    \begin{flalign*}
        &\left(
        \forall x \
        \left(x \in X \implies x \in Y\right)
        \implies
        \forall x, y \
        \left(
        \begin{cases}
            y \in X \\
            x \in y
        \end{cases}
        \implies
        \begin{cases}
            y \in Y \\
            x \in y
        \end{cases}
        \right)
        \right)
        \implies \\
        &\left(
        \forall x \
        \left(x \in X \implies x \in Y\right)
        \implies
        \forall x \
        \left(
        \exists z \
        \begin{cases}
            z \in X \\
            x \in z
        \end{cases}
        \implies
        \exists w \
        \begin{cases}
            w \in Y \\
            x \in w
        \end{cases}
        \right)
        \right)
        \begin{gathered}
            \iff \\
            def \ \cup
        \end{gathered} \\
        &\left(
        \forall x \
        \left(x \in X \implies x \in Y\right)
        \implies
        \forall x \
        \left(x \in \cup X \implies x \in \cup Y\right)
        \right)
        \begin{gathered}
            \iff \\
            def \ \subseteq
        \end{gathered} \\
        &\left(X \subseteq Y \implies \cup X \subseteq \cup Y\right)
    \end{flalign*}
\end{document}
