\documentclass[oneside]{book}

\usepackage[utf8]{inputenc}
\usepackage[T2A]{fontenc}
\usepackage[russian]{babel}
\usepackage[left = 0.3\textwidth, right = 0.3\textwidth]{geometry}
\usepackage{parskip}
\usepackage[fleqn]{amsmath}
\usepackage{mathtools}
\usepackage{amsfonts}
\usepackage{amssymb}
\usepackage{graphicx}
\usepackage{hyperref}
\usepackage{bookmark}
\usepackage{textcomp}
\usepackage{tikz}
\usetikzlibrary{graphs, positioning}

\setlength{\parskip}{0.03\textheight}

\graphicspath{{images/}}

\newcommand{\set}[1]{\left\{#1\right\}}

\title{Трансляторы: \\ упражнения}
\date{\today}
\author{WinstonMDP}

\begin{document}
    \maketitle

    \section{2.2.1}
    Рассмотрим контекстно-свободную грамматику
    \begin{flalign*}
        S \rightarrow S S + \mid S S * \mid a
    \end{flalign*}
    \begin{enumerate}
        \item Покажите, как данная грамматика генерирует строку $ a a + a * $.
        \item Постройте дерево разбора для данной грамматики.
        \item Какой язык генерирует данная грамматика? Обоснуйте свой ответ.
    \end{enumerate}

    \subsection*{a, b}
    \begin{tikzpicture}
        \node (l_1) {a};
        \node (l_2) [right=of l_1] {a};
        \node (l_3) [right=of l_2] {$ + $};
        \node (l_4) [right=of l_3] {a};
        \node (l_5) [right=of l_4] {$ * $};
        \node (n_1) [above=of l_1] {S};
        \node (n_2) [above=of l_2] {S};
        \node (n_3) [above=of l_4] {S};
        \node (n_4) [above=of n_2] {S};
        \node (n_5) [above right=of n_4] {S};
        \graph {
            (l_1) -- (n_1);
            (l_2) -- (n_2);
            (l_3) -- (n_4);
            (l_4) -- (n_3);
            (l_5) -- (n_5);
            (n_1) -- (n_4);
            (n_2) -- (n_4);
            (n_3) -- (n_5);
            (n_4) -- (n_5);
        };
    \end{tikzpicture}

    \subsection*{c}
    Постфиксных операций.

    \section{2.2.2}
    Какой язык генерируется каждой из следующих грамматик?
    В каждом случае обоснуйте свой ответ.
    \subsection*{а}
    \begin{flalign*}
        S \rightarrow 0 S 1 \mid 0 1
    \end{flalign*}

    Язык строк из чётного количества единиц и нулей, где слева от середины нули и
    их количество равно количеству единиц справа от середины, справа от середины
    только единицы.

    \subsection*{б}
    \begin{flalign*}
        S \rightarrow + S S \mid - S S \mid a
    \end{flalign*}

    Язык префиксной арифметики.

    \subsection*{в}
    \begin{flalign*}
        S \rightarrow S (S) S \mid \varepsilon
    \end{flalign*}

    Язык правильных скобочных последовательностей.

    \subsection*{г}
    \begin{flalign*}
        S \rightarrow a S b S \mid b S a S \mid \varepsilon
    \end{flalign*}

    Язык комбинаций $ a $ и $ b $ чётной длины.

    \subsection*{д}
    \begin{flalign*}
        S \rightarrow a \mid S + S \mid S S \mid S * \mid (S)
    \end{flalign*}

    Язык с инфиксным сложением и с постфиксным умножением.

    \section{2.2.3}
    Какие из грамматик в упражнении 2.2.2 неоднозначны?

    г: \\
    \begin{tikzpicture}
        \node (n_1) {a};
        \node (n_2) [right=of n_1] {b};
        \node (n_3) [right=of n_2] {a};
        \node (n_4) [right=of n_3] {b};
        \node (n_5) [right=of n_4] {a};
        \node (n_6) [right=of n_5] {b};
        \node (n_7) [above=of n_3] {S};
        \node (n_8) [above=of n_7] {S};
        \node (n_9) [above=of n_8] {S};
        \graph {
            (n_1) -- (n_9);
            (n_3) -- (n_7);
            (n_2) -- (n_8);
            (n_4) -- (n_7);
            (n_5) -- (n_8);
            (n_6) -- (n_9);
            (n_7) -- (n_8);
            (n_8) -- (n_9);
        };
    \end{tikzpicture} \\
    \begin{tikzpicture}
        \node (n_1) {a};
        \node (n_2) [right=of n_1] {b};
        \node (n_3) [right=of n_2] {a};
        \node (n_4) [right=of n_3] {b};
        \node (n_5) [right=of n_4] {a};
        \node (n_6) [right=of n_5] {b};
        \node (n_7) [above=of n_3] {S};
        \node (n_8) [above=of n_5] {S};
        \node (n_9) [above right=of n_7] {S};
        \graph {
            (n_1) -- (n_9);
            (n_2) -- (n_7);
            (n_3) -- (n_7);
            (n_4) -- (n_9);
            (n_5) -- (n_8);
            (n_6) -- (n_8);
            (n_7) -- (n_9);
            (n_8) -- (n_9);
        };
    \end{tikzpicture}

    д: например, строка $ a + a * $.

    \section{2.2.4}
    Постройте однозначные контекстно-свободные грамматики для каждого
    из следующих языков. В каждом случае покажите корректность вашей грамматики.

    \subsection*{а}
    Арифметические выражения в постфиксной записи.

    \begin{flalign*}
        S \rightarrow a \mid S S + \mid S S *
    \end{flalign*}

    \subsection*{б}
    Левоассоциативный список идентификаторов, разделённых запятыми.

    \begin{flalign*}
        S \rightarrow a \mid a, S
    \end{flalign*}

    \subsection*{в}
    Правоассоциативный список идентификаторов, разделённых запятыми.

    \begin{flalign*}
        S \rightarrow a \mid S, a
    \end{flalign*}

    \subsection*{г}
    Арифметические выражения, состоящие из целых чисел и идентификаторов с
    четырьмя бинарными операторами $ +, *, -, / $.

    \begin{flalign*}
        &O \rightarrow Z \mid (S) \\
        &P \rightarrow P O \mid P / O \mid O \\
        &S \rightarrow S + P \mid S - P \mid P
    \end{flalign*}

    \subsection*{д}
    Добавьте унарные "плюс"{ }и "минус"{ }к арифметическим операторам из г.

    \begin{flalign*}
        &O \rightarrow Z \mid (S) \\
        &U \rightarrow -O \mid +O \\
        &R \rightarrow (U) \mid O \\
        &P \rightarrow P R \mid P / R \mid R \\
        &S \rightarrow S + P \mid S - P \mid P
    \end{flalign*}

    \begin{flalign*}
        &O \rightarrow Z \mid (R) \\
        &P \rightarrow P O \mid P / O \mid O \\
        &S \rightarrow S + P \mid S - P \mid P \\
        &R \rightarrow -S \mid +S \mid S \\
        &\bot
    \end{flalign*}

    \section{по мотивам 2.3.2}
    Постройте схему синтаксически управляемой трансляции, которая транслирует
    арифметические выражения из постфиксной записи в инфиксную.

    \begin{flalign*}
        &right
        \rightarrow \\
        &[print \ -(] \ sum \ [print \ )] \ -
        \mid \\
        &[print \ +(] \ sum \ [print \ )] \ +
        \mid \\
        &[print \ -]Z \ -
        \mid \\
        &[print \ +]Z \ + \\
        &sum
        \rightarrow \\
        &[print \ (] \ sum \ [print \ )] \ right
        \mid \\
        &Z \ right
    \end{flalign*}
\end{document}
