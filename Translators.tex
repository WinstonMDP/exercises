\documentclass[oneside]{book}

\usepackage[utf8]{inputenc}
\usepackage[T2A]{fontenc}
\usepackage[russian]{babel}
\usepackage[left = 0.3\textwidth, right = 0.3\textwidth]{geometry}
\usepackage{parskip}
\usepackage[fleqn]{amsmath}
\usepackage{mathtools}
\usepackage{amsfonts}
\usepackage{amssymb}
\usepackage{graphicx}
\usepackage{hyperref}
\usepackage{bookmark}
\usepackage{textcomp}
\usepackage{tikz}
\usetikzlibrary{graphs, positioning}

\setlength{\parskip}{0.03\textheight}

\graphicspath{{images/}}

\hypersetup{
    colorlinks,
    citecolor=black,
    filecolor=black,
    linkcolor=black,
    urlcolor=black
}

\newcommand{\meta}[1]{\text{<}#1\text{>}}
\newcommand{\set}[1]{\left\{#1\right\}}

\title{Трансляторы: \\ упражнения}
\date{\today}
\author{WinstonMDP}

\begin{document}
    \maketitle

    \section{2.2.1}
    Рассмотрим контекстно-свободную грамматику
    \begin{flalign*}
        S \rightarrow S S + \mid S S * \mid a
    \end{flalign*}
    \begin{enumerate}
        \item Покажите, как данная грамматика генерирует строку $ a a + a * $.
        \item Постройте дерево разбора для данной грамматики.
        \item Какой язык генерирует данная грамматика? Обоснуйте свой ответ.
    \end{enumerate}

    \subsection*{a, b}
    \begin{tikzpicture}
        \node (l_1) {a};
        \node (l_2) [right=of l_1] {a};
        \node (l_3) [right=of l_2] {$ + $};
        \node (l_4) [right=of l_3] {a};
        \node (l_5) [right=of l_4] {$ * $};
        \node (n_1) [above=of l_1] {S};
        \node (n_2) [above=of l_2] {S};
        \node (n_3) [above=of l_4] {S};
        \node (n_4) [above=of n_2] {S};
        \node (n_5) [above right=of n_4] {S};
        \graph {
            (l_1) -- (n_1);
            (l_2) -- (n_2);
            (l_3) -- (n_4);
            (l_4) -- (n_3);
            (l_5) -- (n_5);
            (n_1) -- (n_4);
            (n_2) -- (n_4);
            (n_3) -- (n_5);
            (n_4) -- (n_5);
        };
    \end{tikzpicture}

    \subsection*{c}
    Правосторонних операций (как-то так).

    \section*{2.2.2}
    Какой язык генерируется каждой из следующих грамматик?
    В каждом случае обоснуйте свой ответ.
    \begin{enumerate}
        \item $ S \rightarrow 0 S 1 \mid 0 1 $.
        \item $ S \rightarrow + S S \mid - S S \mid a $.
        \item $ S \rightarrow S (S) S \mid \varepsilon $.
    \end{enumerate}
\end{document}
